\documentclass[brudnopis]{xmgr}
% Jeśli nowe rozdziały mają się zaczynać na stronach
% nieparzystych:
%\documentclass[openright]{xmgr}

%\defaultfontfeatures{Scale=MatchLowercase}
%\setmainfont[Numbers=OldStyle,Ligatures=TeX]{Minion Pro}
%\setsansfont[Numbers=OldStyle,Ligatures=TeX]{Myriad Pro}
% for fontspec version < 2.0
\setmainfont[Numbers=OldStyle,Mapping=tex-text]{Minion Pro}
\setsansfont[Numbers=OldStyle,Mapping=tex-text]{Myriad Pro}
%\setmonofont[Scale=0.75]{Monaco}

% Opcjonalnie identyfikator dokumentu
% drukowany tylko z włączoną opcją 'brudnopis':
\wersja   {wersja wstępna [\ymdtoday]}

\author   {Czerwińska Agnieszka}
\nralbumu {206\,314}
\email    {czewinska.agnieszka.lucja@gmail.com}

\author   {Maciejewski Michał}
\nralbumu {206\,316}
\email    {michal.maciejewski92@gmail.coml}

\author   {Żelazek Mateusz}
\nralbumu {204\,608}
\email {mat.zelazek@gmail.com}

\title    {Unity - narzędzie dla twórców gier}
\date     {2016}
\miejsce  {Gdańsk}

\opiekun  {dr W. Bzyl}

% dodatkowe polecenia
%\renewcommand{\filename}[1]{\texttt{#1}}
%\definecolor{stress}{cmyk}{0,1,0.13,0} % RubineRed
%\definecolor{topic}{cmyk}{0.98,0.13,0,0.43} % MidnightBlue

\begin{document}

% streszczenie
\begin{abstract}
  Aczkolwiek w~ciągu ostatnich lat skład tekstu wspomagany komputerowo
  całkowicie wyeliminował stosowanie tradycyjnych technik drukarskich,
  to podobny proces w~przypadku publikacji elektronicznych czyli
  publikacji, które w~ogóle nie wykorzystują papieru, a~nośnikem
  informacji staje się ekran komputera nie jest obserwowany.
\end{abstract}

% słowa kluczowe
\keywords{SGML,
 XML,
 XSL,
 dokumenty elektroniczne,
 dokumenty strukturalne}

% tytuł i spis treści
\maketitle

% wstęp
\introduction

Wraz z rosnącym zainteresowaniem grami wzrosło też zainteresowanie procesem ich tworzenia. W związku z tym powstało wiele nowych srodowisk i narzędzi ułatwiających ten proces zarówno początkującym jak i zaawansowanym użytkownikom. 

W dzisiejszych czasach tworzenie gier nie wymaga już posiadania nakładów finansowych, ani nawet doswiadczonego zespołu. Platformy do tworzenia gier rozwinęły się do stopnia umożliwiającego łatwy start każdemu początkującemu twórcy gier.  

Różnorodnosć narzędzi dostępnych na rynku jest ogromna i może stanowić problem w wyborze swojego srodowiska. Chcąc ułatwić ten wybór zdecydowalimy się przetestować jeden z nich. Zaintrygowani reklamą producenta o elastycznosci, łatwosci w obsłudze i kompletnosci jednego z dostępnych na rynku  produktów nie moglismy oprzeć się pokusie zweryfikowania tych obietnic. Wybór ten padł na popularny obecnie silnik Unity.

Podczas testów sprawdzalismy cechy, które uznalismy za uniwersalnie dobre dla przystępnego srodowiska. Zwrócilismy uwagę m.in na przejrzystosć interfejsu, aktywnosć społecznosci, oraz wydajnosc, płynnosć i szybkosć działania programu.

\chapter{Wprowadzenie do Unity}

opis

\section{Konfiguracja}
\section{Interfejs}
\section{Języki programowania}
\section{Edycja i kompilacja}

% załączniki (opcjonalnie):
\appendix
\chapter{Tytuł załącznika jeden}

Treść załącznika jeden.

\chapter{Tytuł załącznika dwa}

Treść załącznika dwa.

% literatura (obowiązkowo):
\bibliographystyle{unsrt}
\bibliography{xml}

% spis tabel (jeżeli jest potrzebny):
\listoftables

% spis rysunków (jeżeli jest potrzebny):
\listoffigures

\oswiadczenie

\end{document}
